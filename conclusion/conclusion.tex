\chapter{Conclusion}
La motivation de ce mémoire était de présenter de façon rigoureuse la théorie des polynômes orthogonaux. Dans ce but, nous avons introduit le sujet par l'intermédiaire de la méthode d'orthogonalisation de Gram-Schmidt. Nous avons ensuite décortiqué certaines propriétés de base desdits polynômes, et finalement, nous avons abordé les applications.
\\Ceci est sans parler des sujets que nous n'avons pas abordé du tout, mentionnons la théorie des mesures secondaires, ou encore l'utilisation d'ensembles de polynômes dits « biorthogonaux » en codage et décodage de signaux.
\\Les objets en apparence simples mais pleins de potentiels que sont les polynômes sont encore étudiés sous plusieurs angles. Il est donc justifié de croire que la théorie continuera encore de se développer et de donner des applications intéressantes.