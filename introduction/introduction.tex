\chapter{Introduction}
Les polynômes orthogonaux sont un sujet d'étude pour les mathématiciens depuis des décennies. \`{A} titre d'exemple, Adrien-Marie Legendre en était arrivé dès le XIX\up{e} à considérer la suite de polynômes auxquelles son nom est maintenant associé (les polynômes de Legendre) dans le cadre de ses calculs concernant la mécanique céleste. Depuis cette époque jusque aujourd'hui, la théorie concernant ces polynômes n'a cessé de se développer en précision et aussi en importance avec d'autres applications dans différents domaines. En effet les polynômes orthogonaux sont utiles en physique mathématiques lors de la résolution de certaines équations aux dérivées partielles (Laplace, Schrödinger) par la méthode de séparation des variables. Aussi, en analyse numérique, avec l'avènement des ordinateurs, ils sont un des outils d'approximation et d'encodage-décodage. 
\\Les familles de polynômes vérifient souvent certaines relations de récurrence et sont aussi solutions d'équations différentielles. Nous présenterons dans un premier temps la théorie générale sur ces polynômes et nous montrerons ensuite quelques propriétés qu'ils vérifient. Dans la deuxième partie on s'intéressera à des familles particulières de polynômes orthogonaux classiques à savoir les polynômes de Legendre, de Tchebychev, de Laguerre et de Hermite. Pour terminer, nous aborderons quelques applications importantes des polynômes orthogonaux dans le domaine scientifique.
