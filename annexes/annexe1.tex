\chapter{Obtention des polynômes orthogonaux classiques par le procédé d'orthogonalisation de Gram-Schmidt}
Nous donnons ci-dessous différents choix de l'intervalle $I$et de la fonction densité $w$ conduisant à des familles de polynômes orthogonaux classiques connues.

\paragraph{Polynômes de Legendre}
$ $\\Les polynômes de Legendre $P_n$ sont obtenus en considérant $$I=[-1,1] \text{ et } w(x)=1,\ \forall x\in [-1,1].$$
Les polynômes de Legendre $P_n$ vérifient la relation: $$ P_n(x)=\dfrac{d^n}{dx^n}(x^2-1)^n,\ \forall\ x\in [-1,1].$$
\paragraph{Polynômes de Laguerre}
$ $\\Les polynômes de Laguerre $L_n$ sont obtenus en choisissant $$I=[0,+\infty[ \text{ et } w(x)=e^{-x}, \ \forall \ x\in [0,+\infty[$$
Les polynômes de Laguerre $L_n$ vérifient la relation:$$ L_n(x)=(-1)^n e^x \frac{d^n}{dx^n}(e^{-x}x^n), \ \forall x\in [0,+\infty[.$$
\paragraph{Polynômes de Hermite}
$ $\\Les polynômes de Hermite $H_n$ sont obtenus en considérant $$I=\R \text{ et } w(x)=e^{-x^2},\ \forall \ x \in \R.$$
Les polynômes de Hermite $H_n$ vérifient la relation:
$$ H_n(x)=(-1)^n e^{x^2}\dfrac{d^n}{dx^n}(e^{-x^2})\ \forall x\in \R.$$
\paragraph{Polynômes de Tchebychev}
$ $\\Les polynômes de Tchebychev $T_n$ sont obtenus en choisissant $$ I=]-1,1[ \text{ et } w(x)=\dfrac{1}{\sqrt{1-x^2}}, \ \forall \ x\in ]-1,1[.$$
Les polynômes de Tchebychev $T_n$ vérifient la relation:$$ T_n(x)=\cos(n \arccos(x)),\ \forall x\in ]-1,1[.$$